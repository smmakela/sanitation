\documentclass[10pt,a4paper]{article}
\usepackage[utf8]{inputenc}
\usepackage{amsmath}
\usepackage{amsfonts}
\usepackage{amssymb, setspace}
\usepackage{geometry}
\geometry{left=1in, right=1in}
\onehalfspacing
\author{Susanna Makela}
\title{Causal Impact of Sanitation on Child Health in India}
\begin{document}
\maketitle

Diarrheal diseases are a major source of disease burden in India, accounting for an estimated 8\% of all deaths among children under 5 in India in 2015. % IHME GBD 2015
Children who suffer from chronic diarrheal diseases are more susceptible to malnutrition, which affects nearly half of children under 5. They are also at greater risk for opportunistic infections like pneumonia, another major cause of death (lower respiratory tract infections constitute over 10\% of under-5 deaths). % UNICEF http://www.unicef.in/Whatwedo/10/Stunting and IHME GBD 2015

The Government of India has undertaken major projects to improve access to adequate sanitation facilities, but uptake remains low, particularly in rural areas. %NEED CITATION
Studies have demonstrated the existence of positive externalities to child health from improved community-level sanitation, but the interaction of household- and community-level sanitation remains unclear. %Geruso and Spears 2015
Understanding the interplay between household- and community-level sanitation and disentangling their causal effects on the incidence of diarrheal diseases is an important step in designing effective sanitation policy interventions.

The data I intend to use will come from some combination of the National Family Health Survey (NFHS) and the District Level Health Survey (DLHS). Both are nationally representative household surveys, with the NFHS designed to allow for state-level estimates of key demographic and health indicators, while the DLHS was designed to allow for district-level estimates. Data are available from four rounds of each survey. The NFHS was conducted in 1992-1993, 1998-1999, 2005-2006, and 2015-2016, and the DLHS in 1998-1999, 2002-2004, 2007-2008, and 2012-2014.

The units of observation are children for whom we have information on recent diarrheal episodes. Generally, this means surviving children born to interviewed women in the three years prior to the survey date. The treatment variables are a binary indicator for whether a household has access to an improved sanitation facility and a continuous measure of the proportion of sampled households in the community (defined as a primary sampling unit as used in the sampling design) with access to improved sanitation. The outcome variable is whether the child had diarrhea in the two weeks prior to the survey date. 

\end{document}