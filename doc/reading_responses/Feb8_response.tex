\documentclass[10pt,a4paper]{article}
\usepackage[utf8]{inputenc}
\usepackage{amsmath}
\usepackage{amsfonts}
\usepackage{amssymb}
\usepackage{setspace}
%\usepackage[left=2cm,right=2cm,top=2cm,bottom=2cm]{geometry}
\onehalfspacing
\author{Susanna Makela}
\title{Reading Response}
\onehalfspacing
\begin{document}
\maketitle

As Pearl states in section 3, in drawing a causal graph, we are implicitly making claims about which relationships are not causal. This is similar to graphical models, where the lack of an arrow between two vertices implies that, conditional on the parents of one, the two vertices are independent. I suppose making such strong claims, in one form or another, is necessary for doing causal inference, but I always get caught up in drawing the causal graph in the first place, especially in the context of public health, which is what my project focuses on. Every time I've seen a causal graph in the public health context, it looks like a giant web where nearly everything is connected to everything else. A survey like the National Family and Health Survey contains hundreds of measurements on households, women, some of their husbands, and children. Constructing a graphical model in this context seems to me to be almost \textit{more} difficult because of the large amount of data available; it becomes tempting to create ever-larger causal graphs since we have measurements on so many things.

Of course, claiming that something is difficult is not an excuse to not do it. I just wonder if there is a process for building the causal graph that keeps you from creating an unending web of seemingly-plausible dependencies.

\end{document}